\begin{center}
    \vspace*{0.8cm}
    {\huge\bf STÖTAR \& ÖVERFÖRING}\\
    \vspace{0.5cm}
    \large{Rapport gällande faktorer som påverkar elasticitet, energibevaring och rörelsemängdsmomentsöverföring i stötar.}\\
\end{center}
\vspace{1.0cm}

\begin{center}
Oskar Jonsson (cid: osjon) och Vidar Petersson (cid: vidarp)

Program: Teknisk Fysik. 

Kurs: Experimentell fysik 1 - mätteknik, TIF083, del A.
\end{center}
\vspace{1.0cm}

\centerline{\bf Sammandrag}
\noindent Denna rapport undersöker kollisioner i en och två dimensioner med fokus på bevarande och fördelning av energi och rörelsemängd samt rörelsemängdsöverföring. I endimensionella kollisioner av ryttare på en luftskena studeras sambandet mellan stötkoefficienten $e$ och den relativa kollisionshastigheten $v_{\text{rel}}$ för olika material. Resultaten visar att $e$ minskar när $v_{\text{rel}}$ ökar, med olika karaktär beroende på material. I tvådimensionella kollisioner för puckar på ett luftbord analyseras påverkan av kollisionspunkten $d$ mellan två puckar för fördelningen av kinetisk energi, rörelsemängd och rörelsemängdsmomentsöverföring. Resultaten avslöjar att fördelningen av kinetisk energi förändras avsevärt med $d$, med en ökning av andelen rotationsenergi när $d$ ökar. Dessutom analyseras riktningen på rörelsemängden efter kollisionen i förhållande till kollisionspunkt $d$. Resultaten ger värdefulla insikter om energi- och rörelsemängdsbevarande i olika dimensioner och har implikationer för tillämpningar som kollisionssimuleringar och beräkningar av planetsystem. Rapporten belyser även vikten av att beakta faktorer som friktion i riktiga kollisionssituationer, vilket kan avvika från idealiska teoretiska modeller.

{\bf Nyckelord:} kollision, stötkoefficient, energibevaring, elasticitetsgräns


\centerline{\bf Abstract}
\noindent This report investigates collisions in one and two dimensions with a focus on the conservation and distribution of energy and momentum, as well as momentum transfer. In one-dimensional collisions of riders on an air track, the relationship between the coefficient of restitution $e$ and the relative collision velocity $v_{\text{rel}}$ for different materials is studied. The results show that $e$ decreases as $v_{\text{rel}}$ increases, with different characteristics depending on the material. In two-dimensional collisions of pucks on an air table, the impact of the collision point $d$ between two pucks on the distribution of kinetic energy, momentum, and angular momentum transfer is analyzed. The results reveal that the distribution of kinetic energy changes significantly with $d$, with an increase in the proportion of rotational energy as $d$ increases. Additionally, the direction of momentum after the collision in relation to the collision point $d$ is analyzed. The results provide valuable insights into the conservation of energy and momentum in different dimensions and have implications for applications such as collision simulations and calculations of planetary systems. The report also highlights the importance of considering factors like friction in real collision situations, which may deviate from ideal theoretical models.

{\bf Key words:} collision, coefficient of restitution, conservation of energy, elastic limit
