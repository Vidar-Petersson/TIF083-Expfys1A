Sammanfattningsvis har detta arbete undersökt stötar i både en och två dimensioner med avseende på energi- och rörelsemängdsbevaring samt fördelningen av energi och rörelsemängd. I endimensionella stötar visade resultaten att stötkoefficienten \(e\) minskar med ökad relativ hastighet \(v_{\text{rel}}\) och att detta samband varierar beroende på materialet som används. Gummi visade sig ha en linjär relation mellan \(e\) och \(v_{\text{rel}}\), medan metallen uppvisade ett potenssamband. Detta kan troligen förklaras av skillnader i elasticitetsgräns mellan materialen.

I två-dimensionella stötar visade resultaten att fördelningen av kinetisk energi förändras beroende av kollisionspunkten \(d\). Rotationsenergins andel ökar med ökad \(d\), medan translationsenergins andel minskar. Detta samband kan troligen kopplas till friktion vid kollisionsögonblicket och resulterande tangentiella impulser. Rörelsemängden i $x'$- och $y'$-led för puck B påverkas också av \(d\), och det observerades att rotationsrörelsen i puck A påverkar rörelsemängden i puck B efter kollisionen. Dessutom ökar det absoluta rörelse-mängdsmomentet för puck B med ökad \(d\).

Dessa resultat ger insikter om hur energi- och rörelsemängdsbevaring fungerar i olika dimensioner och hur fördelningen av energi och rörelsemängd påverkas av kollisionsparametrar. Denna kunskap kan vara värdefull inom olika fysikaliska och tekniska områden, inklusive krocksimulationer och beräkning av planetsystem. Det är viktigt att notera att friktionens påverkan på kollisioner i praktiken kan vara betydande och kan ge upphov till avvikelser från ideala teoretiska modeller. 

Utvecklingsmöjligheterna för detta arbete ligger främst i att bättre förklara orsaken till de observerade sambanden och resultaten teoretiskt. I framtida undersökningar för del 1 bör fler ingående parametrar observeras. Exempelvis kan energiförluster i form av värme och ljud mätas. I del 2 kan en skjutanording användas för att säkerställa att kollisionshastighet och rotation innan stöten hålls konstant när $d$ varieras.

%