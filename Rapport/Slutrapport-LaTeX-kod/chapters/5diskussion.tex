Baserat på de redovisade resultaten i del 1 kan det tydligt konstateras att gummi överlag är ett mer elastiskt material jämfört med metall. Samtidigt minskar elasticitetskoefficienten \(e\) när den relativa hastigheten \(v_{\text{rel}}\) ökar för båda materialen. Med höga \(R^2\)-värden på 0,96 och 0,93 kan vi också påvisa att mätvärdena i stor utsträckning följer de anpassade funktionerna från regressionsanalysen. Det är intressant att notera hur sambandet varierar beroende på materialet. Det är svårt att entydigt förklara varför gummi uppvisar en linjär relation medan metallen följer en potensrelation enbart utifrån den presenterade teorin. En möjlig förklaring är att gummibandet har en betydligt högre elasticitetsgräns än metallen. Det innebär att det befinner sig inom sitt elastiska område inom det undersökta intervallet av \(v_{\text{rel}}\) främst förlorar energi till värme istället för att undergå plastisk deformation som metallen gör. Detta resonemang stärks av den tydliga utplaningen av \(e\) för metallen vid \(v_{\text{rel}} > \SI{0.6}{m/s}\). Det verkar som att materialet når sin elasticitetsgräns inom detta interval.

I del 2 undersöks flera aspekter av den tvådimensionella kollisionen med fokus på den initiala frågeställningen. Från mätdatan om fördelningen av kinetisk energi framgår tydligt att rotations- och translationsenergi korrelerar med \(d\). Ju längre ut mot kanten puckarna träffar varandra, desto större blir rotationsenergins andel av den totala kinetiska energin. Mätvärdena överensstämmer i hög grad med de anpassade funktionerna och ger höga \(R^2\)-värden på 0,84 och 0,86. Det observerade sambandet går dock emot det teoretiska idealfallet där impulser från puckkollisioner enbart påverkar translation eftersom impulsen skall vara riktad vinkelrätt mot kontaktytan genom masscentrum. Ytterliggare en intressant observation är att rotationsenergins andel inte överstiger cirka \(30\%\) även när \(d\) närmar sig 1. En högst trolig förklaring är förekomsten av friktion mellan kontaktytorna vid kollisionsögonblicket, vilket ger upphov till en tangentiell kraft och därmed en impuls parallellt med kontaktytan och således även ett impulsmoment. Detta impulsmoment ökar rimligen med \(d\) eftersom puck A:s rörelsemängdsvektor har en större komponent i tangentiell riktning till kollisionsytan då. Impulsmomenten förklarar sambandet med rotationsenergin, medan den tangentiella impulsen förklarar varför translationsenergin fortfarande utgör cirka \(70\%\) av den totala energin när den vinkelräta impulsen minskar för stora \(d\).

Det är också värt att notera att inte bara rotationsenergins andel ökar med \(d\), utan även det absoluta rörelsemängdsmomentet för puck B, som kan ses i figur \ref{fig:lb-label}. Dock har korrelationen ett lägre $R^2$-värde på 0,51. Svagheten i korrelationen kan möjligen förklaras av det faktum att rotationshastigheten för puck B också är beroende av rotationshastigheten för puck A innan kollisionen. Denna parameter kunde inte hållas konstant och kan sannolikt vara ansvarig för bruset i mätdatan.

Ytterligare en aspekt som analyseras är riktningen på rörelsemängdsvektorn för puck B efter stöt i förhållande till \(d\). Figur \ref{fig:px} och \ref{fig:py} visar hur puck B:s rörelsemängdsvektor delas upp i en vinkelrät och en parallell komponent i förhållande till puck A:s rörelsemängdsvektor före kollisionen. Parameteren \(p_{x'}\) mäter i grunden hur mycket puck B avviker åt sidan efter kollisionen. Även om korrelationen är svag med $R^2$-värde på 0,25 och 0,35, stödjer det den intuitiva insikten att ju längre från centrum av kollisionen puckarna kolliderar, desto mer avvikelse uppstår i sidled. Svagheten i sambandet kan troligtvis förklaras på samma sätt som för den kinetiska energin, genom impulser som verkar tangentiellt med kontaktytan och då påverkar riktningen.

Angående bevaring av rörelsemängd bekräftar resultatet teorin. I figur \ref{fig:rm} syns ingen tydlig trend och  $R^2$-värdet för regressionsanalysen är 0,05. Differensen av rörelsemängd innan och efter stöt ligger även runt 0, vilket bekräftar Newtons lagar om bevaring för slutna system.

Datainsamlingens riktighet bör även betraktas som god då mätvärdena har en relativt låg felmarginal och att energipåverkan i referensmätningarna från bordet var liten och systematisk vilket gör att påverkan på sambandet blir försumbart.

%Utifrån redovisade resultat i del 1 kan ett tydligt samband konstateras där gummi är ett överlag mer elastiskt material än metall samtidigt som $e$ avtar när $v_{\text{rel}}$ ökar för båda materialen. Med $R^2$-värde på 0,96 respektive 0,93 påvisas även att mätvärdena till hög grad följer de anpassade funktionerna som erhålls från regressionsanalysen. Dock är det intressant hur sambandet skiljer sig åt beroende på material. Anledningen till att gummi ger ett linjärt samband och metall ett potenssamband kan ej förklaras entydigt utifrån den presenterade teorin. Möjlig förklaring är att gummiband har en betydligt högre elasticitetsgräns än metallen vilket innebär att den befinner sig i sitt elastiska område för det undersökta intervallet på $v_{\text{rel}}$. Det linjära avtagandet beror således troligtvis på att gummibandet förlorar energi till värme istället för plastisk deformation som metallen. Detta resonemang styrks av ett en tydlig utplaning av $e$ för metallen kan observeras för $v_{\text{rel}}>0,6$. Bedömt uppnår materialet sin elasticitetsgräns i det intervallet.

%I del 2 undersöks flera aspekter av den tvådimensionella stöten utifrån frågeställningen. Från mätdatan om fördelningen av kinetisk energi erhålls inledningsvis ett tydligt samband att rotations- och translationsenergi beror på $d$. Desto längre ut på kanten puckarna träffar varandra ju större blir rotationsenergins andel av den totala kinetiska energin. Mätvärdena följer de anpassade funktionerna med ett relativt högt $R^2$-värde på 0,84 och 0,86. Detta samband motsäger dock det teoretiska idealfallet där impulser från kollisioner av puckar enbart påverkar translation på grund av att impulsen är riktade vinkelrät mot kontaktytan genom masscentrum. En ytterligare intressant observation är att andelen rotationsenergi inte blir större än cirka $30 \%$ även för $d$ nära 1. Högst trolig förklaring är att förekomsten av friktion mellan kontaktytorna i stötögonblicket ger upphov till en tangentiell kraft som introducerar en impuls parallell med kontaktytan och således ett impulsmoment. Impulsen och impulsmomentet ökar rimligen tillsammans med $d$ eftersom puck As rörelsemängdssvektor har en större komposant tangentiellt till kollisionsytan då. Impulsmomenten förklara isåfall sambandet med rotationsenergin och den tangentiella impulsen förklarar varför translationsenergin fortfarande utgör cirka 70 \% av totala energin när den vinkelräta impulsen blir mindre för stora $d$.

%Observation av att storheterna inte helt är bevarade enligt tidigare presenterade uttryck kommer troligtvis att göras på grund av det icke-ideala system som försöken kommer att göras i. På grund av att faktorer som både kan tillföra och tillfrånta energi till systemet kommer vara närvarande kommer till exempel stötkoeffecientens uträkning inte bli exakt. Även lufttillförseln i experimenten kan bli svår att reglera så att den skapar ett system där friktionen blir så nära noll för att behålla det slutet och beroende på alla faktorers noggrannhet samt reglerbarhet kommer överföringen divergera från 100\%.

%I den första uppställningen del 1, där relativ hastighet varieras mellan ryttarna i en en-dimensionell stöt, förväntas stötkoefficienten enligt presenterad teori idealiskt sett inte att påverkas. Dock kommer eventuella experimentella faktorer som ökade inre friktioner, rörelse i $\hat z$-led, energiförluster till ljud och värme i gummibandet påverka resultatet vid högre hastigheter. Vid lägre hastigheter kan det däremot bli intressant att studera ifall luftströmmar från skenan gör det möjligt för $e<1$ att observeras.

%I del 2 förväntas ett större $d$ enligt presenterad teori i ett idealt fall enbart påverka fördelningen mellan rörelsemängd i $\hat x$ och $\hat y$-led vid stöt. Detta på grund av att idealfallet inte tar hänsyn till friktion mellan puckarnas kontaktytor. I och med att puckarna är cirkelformade uppstår enbart radiella impulser som i sin tur enbart påverkar masscentrums translation. Genom att däremot ta hänsyn till tangentiell friktion i stöten bör stora $d$ ge upphov till större tangentiella krafter vilket i sin tur resulterar i ett impulsmoment. Det bör leda till en större rörelsemängdsmomentsöverföring som naturligt kommer bli en större andel av den ursprungliga rörelsemängden. Liknande experimentella faktorer som påverkar del 1 kommer rimligtvis även vara relevanta i del 2. Därför skall åtgärder tas för att optimera dessa faktorer så bra som möjligt för att försöka bibehålla det slutna systemet och minimera yttre faktorers påverkan.

%Resultaten av dessa experiment har potential att tillämpas inom olika fysikaliska och tekniska områden. Exempelvis kan insikterna från studien vara värdefulla både för att förbättra krocktestning i bilindustrin eller förstå kollisioner mellan himlakroppar. Detta genom att bättre förstå hur olika faktorer påverkar stötar och energiöverföring.