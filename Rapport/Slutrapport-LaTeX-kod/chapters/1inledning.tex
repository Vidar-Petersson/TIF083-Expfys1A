Stöt, eller kollision, är ett fenomen inom klassisk mekanik där kroppar möts och växelverkar under kort tid. Dessa stötar sker i allt från galaxer ner till den mikroskopiska nivån. Därför är det intressant att undersöka hur olika ingående faktorer påverkar stötars egenskaper. Detta arbete syftar speciellt till att presentera hur bevaring och överföring för stötar i olika dimensioner påverkas.

Det övergripande målet med detta arbete är att analysera energi- och rörelsemängds-bevaring för stötar i en och två dimensioner samt rörelsemängdsmomentöverföring i två dimensioner. För att uppfylla detta mål koncentrerar sig detta arbete på att besvara följande två konkretiserade frågeställningar:

\begin{itemize}
    \item Hur påverkar relativ kollisionshastighet för olika material stötkoefficienten i en endimensionell stöt? %Rörelsemängd
    \item Hur påverkar kollisionspunkten mellan två puckar fördelningen av rörelsemängd, translations- och rotationsenergi samt rörelsemängdsmomentsöverföring i en två-dimensionell stöt?
\end{itemize}

Genom en kombination av teoretisk analys, experimentell datainhämtning och dataanalys syftar vi till att ge insikter av betydelse inom ovanstående faktorer som styr kollisioner. Dessa insikter kommer att vara av betydelse för tillämpningar inom olika fysikaliska och tekniska områden som till exempel krocksimulationer och beräkning av planetsystem.

\begin{comment}
    Här beskriver du vad laborationen går ut på, helt enkelt vad den syftar till att göra. Detta gör du för att 
(i) förbereda läsarna på resten av rapporten på ett bra sätt (ii) hjälpa läsarna att förstå på vilket sätt de 
bör läsa rapporten (så att de läser den som du vill/d. v. s. läser den på ”rätt sätt”), och för att (iii) läsarna 
ska kunna använda det som sägs här som ett slags färdriktningsvisare för läsningen av rapporten (läsare 
av den här typen av texter gillar att du skapar, och infriar, förväntningar).   
Skriv  gärna  det  här  avsnittet  i  rapporten  enligt  ett  deduktivt  mönster;  inled  med  något  generellt 
påstående, men naturligtvis ett påstående med direkt koppling till det som ska göras, t. ex. ”Då en kropp 
faller från vila påverkas den i huvudsak av två krafter”, eller något liknande. Succesivt ska texten röra 
sig från det generella till det specifika, d. v. s. syftet med undersökningen/laborationen.  
Syftet anges ibland i instruktionerna till en (lärande)laboration och då är det naturligt att använda något 
liknande den beskrivningen. I ”verklighetens” rapporter finns det sällan instruktioner för en laboration 
och då är det viktigt att man formulerar syftet med laborationen på ett mycket tydligt sätt för läsarna – i 
annat fall riskerar man att de helt missar det som rapporten är tänkt att göra; akademiska texter är texter 
som ARBETAR, texter som GÖR något för läsaren. 
\end{comment}