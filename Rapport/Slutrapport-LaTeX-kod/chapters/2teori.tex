Kollisioner mellan kroppar introducerar överföring av rörelsemängd, rörelsemängdsmoment och energi som är centrala inom den klassiska mekaniken. Dessa fenomen beskriver förändring av kroppars translation och rotation i deras rörelsedimensioner och är bevarade ur ett systemspecifikt perspektiv, det vill säga att dessa kroppars förändring av dessa storheter går att förutspå med hjälp av uttryck härledda från Newtons lagar. 

Rörelsemängd är en vektorstorhet, betecknat $\mathbf{p}$, som är vektorprodukten av en kropps massa \emph{m} och dess hastighetsvektor $\mathbf{v}$. Rörelsemängden är bevarad i ett slutet system om inga yttre impulser påverkar systemet vilket innebär att ett system av flera kroppars rörelsemängd kan adderas med avseende på varje kropps enskilda rörelsemängd enligt \cite{Meriam_Kraige_Bolton_2020} %Rätt ställe?

\begin{equation} \label{rörelsemängd}
    \mathbf{p}= \sum_i m_i\mathbf{v}_i\text.
\end{equation}

Rörelsemängdsmoment är även det en vektorstorhet som beskriver en kropps rotationstillstånd med avseende på dess massfördelning samt vinkelhastighet. Denna storhet beskrivs generellt med uttrycket

\begin{equation} \label{rörelsemängdsmoment}
    \mathbf{L}_{O}   = \bar{\mathbf{I}}\boldsymbol{\omega} + \mathbf{r}_{MC} \times \mathbf{p}
\end{equation}

där $\mathbf{L}_{O}$ är rörelsemängdsmomentet kring en godtycklig punkt i systemet, $\mathbf{r}_{MC}$ är ortsvektorn till kroppens masscentrum och $\mathbf{p}$ är kroppens rörelsemängdsvektor. $\bar{\mathbf{I}}$ är kroppens tröghetsmoment med avseende på masscentrum och $\boldsymbol{\omega}$ är kroppens rotationsvektor med avseende på masscentrum. Rörelsemängdsmoment är likt rörelsemängden bevarad i ett slutet system \cite{Meriam_Kraige_Bolton_2020}. Vid analys av enbart rörelsemängdsmoment kring masscentrum kan den förenklade formeln $\mathbf{L}_{MC} = \bar{\mathbf{I}}\boldsymbol{\omega}$ användas. 

Den totala kinetiska energin för en kropp definieras enligt
 \begin{equation}\label{kinetiskenergi}
     T_{\text{tot}} = T_{\text{trans}}+T_{\text{rot}} = \frac{1}{2}mv^2 + \frac{1}{2}\Bar{I}\omega^2
 \end{equation}
där parametrarna är de storheter som tidigare definierats i ekvation \eqref{rörelsemängd} och \eqref{rörelsemängdsmoment} för en godtycklig kropp fast utan sina vektoregenskaper vilket leder till att $T$ inte har någon riktning utan enbart är en skalär \cite{Meriam_Kraige_Bolton_2020}.

I ett slutet system är energin enbart bevarad i idealfallet. Detta innebär att i en ideal stöt mellan två kroppar är den totala energi för systemet konstant. I praktiken är en stöt däremot oftast inelastisk och en energiförlust sker på grund av exempelvis plastisk deformation, värme eller ljud. Elasticiteten för en endimensionell stöt beskrivs av stötkoefficienten enligt 
\begin{equation} \label{ekv:e}
     e = \frac{v_{\text{rel}}'}{v_{\text{rel}}}
 \end{equation}
där $v_{\text{rel}}'$ är relativa hastigheten omedelbart efter och $v_{\text{rel}}$ omedelbart före kollision. Stöt-koefficient med kvot 1 representerar således en perfekt elastisk stöt. I allmänhet betraktas $e$ för stöt mellan två kroppar som en konstant beroende på deras material \cite{Meriam_Kraige_Bolton_2020}. Dock visar en mer rigorös analys att även kollisionshastighet påverkar $e$ \cite{stöte}. För hårda material bestämmer deras elasticitetsgräns den högsta spänning de tål utan att deformeras plastiskt. För små kollisionshastigheter är spänningarna i materialet under elasticitetsgränsen och således uppträder materialet helt elastiskt i likhet med en fjäder och all energi bevaras. Om spänningarna däremot överstiger elasticitetsgränsen kommer enbart en del av energi bevaras, resten förloras till deformation. I kollisioner är impulsen i materialet proportionerligt med $v$. Teoretiskt konvergerar därför $e$ för alla material i 1 för små kollisionshastigheter \cite{Meriam_Kraige_Bolton_2020}. 




%Genom att studera bevarandet av dessa storheter i ett slutet system kan man med hjälp av de uttryck som presenterats bestämma överföringen av rörelsemängd, rörelsemängdsmoment samt energi för kropparna i systemet. Denna teori är därmed essentiell för att studera stötkoefficienten, fördelningen mellan rörelsemängd och rörelsemängdsmoment samt energibevaringen för två kroppar i ett slutet system.

